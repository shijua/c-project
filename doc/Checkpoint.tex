\documentclass[11pt]{article}

\usepackage{fullpage}

\begin{document}

\title{Interim Checkpoint Report}
\author{Weijun Huang, Lihaomin Qiu, Molan Yang, Yifan Jiang}

\maketitle

\section*{Task Assignment}

\textbf{Group Members}:

\begin{itemize}
  \item Weijun Huang (Leader): Responsible for implementing pipelining, handling input and output, and making files.
  \item Lihaomin Qiu: Tasked with implementing Single Data Transfer and designing the report.
  \item Molan Yang: Assigned to implement Data Processing Instruction and Bitwise Shifts.
  \item Yifan Jiang: Delegated with implementing Branch Instructions and aiding Molan Yang in debugging.
\end{itemize}

\section*{Working Methodology and Reflection}

\subsection*{Areas of Improvement}

Upon reflection, we acknowledged that our initial strategy presented certain challenges. A fundamental issue was our lack of comprehensive understanding of the project specifications at the outset. This resulted in a disparity in task distribution, with some team members completing their code before others. We mitigated this issue by reorganizing our tasks. Team members who had finished their tasks provided assistance to others, facilitating collective debugging and code completion.

\subsection*{Future Adjustments}

To enhance the efficiency of our team in future projects, we have identified key areas requiring improvement:

\begin{enumerate}
  \item \textbf{Communication}: We acknowledge that clear and frequent communication is integral to the success of a team. To bridge communication gaps, we intend to improve our communication channels and establish more regular meetings.
  \item \textbf{Skill Sets and Expertise}: Before task assignment, we aim to evaluate the skills and expertise of the group members. This might necessitate additional training to equip our team members with the required skills.
  \item \textbf{Time Management}: Meeting deadlines is critical in managing complex projects like an emulator. Therefore, we intend to establish clear deadlines, break down tasks into manageable components, and regularly monitor progress to ensure timely completion.
\end{enumerate}

\section*{Development Process}

The emulator project was divided into various sections, each assigned to a different team member based on an initial workload estimation. Weijun Huang, the team leader, handled the input/output and makefile sections.

Post individual code completion, we collaborated during a lab session to work on the interface code, which was then tested with the main emulator program. However, initial testing yielded limited success.

By setting breakpoints and altering the input file, we identified several problems at the interface between the main program and the function. Correcting these interface issues led to the successful passage of some tests.

We prioritized debugging sections that presented more errors in their corresponding generated tests. This led to a swift resolution of bugs in the \textbf{Single Data Transfer} and \textbf{Load Literal} sections.

The \textbf{Arithmetic operations} section posed significant challenges, primarily due to the need to consider 32bit or 64bit result storage and flag setting according to the result.

Ultimately, we were faced with unresolved issues in the \textbf{branch} section. However, after a period of dedicated debugging, we discovered that the problems were associated with both the \textbf{Subs} section and \textbf{zero register}. Eventually, we managed to address and resolve these issues, and passed all the tests.

\section*{Projecting Forward}

\subsection*{Reuse of Emulator}

The emulator is designed to emulate the execution of an assembly program. Given its similarity to the assembly language, the concepts utilized in the emulator will prove beneficial in our assembler project. We can use the emulator as a foundation for the assembler, implementing most features.

Our prior analysis of the A64 instruction set in the emulator project will expedite the assembler's completion, eliminating the need for reanalysis.

\subsection*{Future Challenges}

While we are optimistic about the assembler project, we anticipate certain complexities, particularly the decision between a \textbf{one-pass assembler} and a \textbf{two-pass assembler}.

The group work challenges encountered in the emulator project have prompted us to adjust our task and personnel arrangements for future projects. With these lessons in mind, we are confident of delivering superior performance in our upcoming endeavors.

\end{document}
